\chapter{Einleitung}
\section{Motivation}
\begin{itemize}
	\item Bedeutung der Web Sicherheit
	\item bekannte Hackerangriffe
	\item anderen Sicherheit beibringen
\end{itemize}

\section{Das Projekt SEED}
SEED ist ein Projekt der Universität Syracuse. Dieses Projekt soll einem helfen verschiedene Angriffsszenarien auszuprobieren. Damit alles legal ausgetestet werden kann, ist dieses Projekt in einer virtuellen Maschine. Somit werden alle Seiten, auf denen getestet werden soll auf einen lokalen Speicher in der virtuellen Maschine geleitet, ohne das es für den Angreifer einen unterschied macht. \\
Insgesamt gibt es 6 verschiedene Kategorien von Laboren. Jede Kategorie hat verschiedene Labore, die zu der entsprechenden Kategorie gehören.
\subsubsection{Software Security Labs}
Diese Labore zeigen generelle Angriffsszenarien auf einer Software. Beispielsweise können hier Angriffsszenarien zu den Themen Buffer Overflow und Shellshok durchgespielt werden oder die Environment Variable untersucht werden. Insgesamt gibt es momentan 5 verschiedene Angriffsszenarien zu diesem Thema und 2 weitere Untersuchungen.
\subsubsection{Network Security Labs}
Hier gibt es verschiedene Angriffsszenarien auf der Netzwerkschicht. Dazu zählen zum beispielsweise Angriffsszenarien wie TCP / IP Angriff, Hearthbleed, oder auch zu DNS gibt es verschiedene Angriffsszenarios. Außerdem kann in dieser Kategorie beispielsweise die Firewall sicherheitstechnisch untersucht werden oder es kann auch selbst was implementiert werden in dem Thema VPN.
\subsubsection{Web Security Labs}
In dieser Kategorie gibt es verschiedene Labore zu den Web Securitys. Dabei gibt es zum einen Angriffsszenarien wie Cross-site Scripting und SQL Injection, aber auch Untersuchungen wie das Web Tracking Labor.
\subsubsection{System Security Labs}
In der Kategorie System Security Labs gibt es 3 verschiedene Labore, in denen es eine Untersuchung über der Linux Capability, sowie 2 Implementierungen zu den Themen Role-Based Access Controll, sowie Encrypted File System gibt.
\subsubsection{Cryptography Labs}
In dieser Kategorie gibt es 3 verschiedene Untersuchungen zu dem Themen Secret Key Encryption, One-Way Hash Function und Public-Key Cryptography and PKI.
\subsubsection{Mobile Security Labs}
In dieser Kategorie gibt es zwei verschiedene Labore, die für ein Android Smartphone ausgelegt sind. Ein Labor behandelt das Repackaging und das andere das Rooting eines Android Smartphone, wobei das Rooting eine Untersuchung und das Repackaging ein Angriff ist.
\subsection{Aufbau der Labore}
Der Aufbau der Labore ist einheitlich aufgebaut. Zu jedem Angriffsszenario gibt es eine grobe Erklärung, was dieser Angriff ist. Zusätzlich gibt es eine Aufgabe zu jedem Labor. Diese Aufgaben dauern unterschiedlich lange und sind auch je nach Angriffsszenario unterschiedlich schwer. Die Aufgaben bieten zunächst einen groben Überblick, was das Angriffsszenario ist und was in dieser Aufgabe gemacht wird. Als nächstes folgt eine Anleitung, wie das Labor, also die virtuelle Maschine, einzurichten ist. Manchmal reicht es die virtuelle Maschine laufen zu lassen. Allerdings müssen bei manchen Laboren die Konfiguration des Apache Servers geändert werden, damit dieser HTTP Anfragen nicht an das WWW raus schickt, sondern lokal in der virtuellen Maschine bleibt. Zusätzlich brauchen manche Labore Dateien, die extra in die virtuelle Maschine rein geladen werden müssen beziehungsweise Programme die installiert werden müssen, da sie in der Standard virtuellen Maschine nicht vorhanden sind. Da die Labore zum Beispiel Wissen über Programme oder über Programmiersprachen wie SQL voraussetzen, gibt es bevor das Labor los geht eine Auflistung der Sachen, welches Wissen für das Labor vorhanden sein muss. \\
Nun folgt das eigentliche Labor. Dieses ist so aufgebaut, dass es erst immer eine Erklärung gibt, was die Aufgabe machen soll, und was damit alles angestellt werden kann. Zusätzlich werden Informationen bekannt gegeben, die der Angreifer haben muss. Diese Informationen können zum Beispiel die Anmeldedaten einer Person sein um in die Anwendung rein zu kommen, oder auch der generelle Aufbau der Webseite. Anschließend folgt eine Erklärung was in diesem Labor gemacht werden muss. Diese Erklärungen können zum Teil sehr detailliert sein, sodass nur die Anweisungen befolgt werden müssen und kein beziehungsweise wenig eigenes Wissen hinzugefügt werden muss. Je mehr Aufgaben in dem Labor gemacht werden, desto schwieriger werden die Aufgaben. Somit muss besonders zum Ende hin mehr eigenes Wissen hinzugefügt werden. 
\subsection{Erfahrung zu dem Projekt SEED}
\begin{itemize}
	\item meine Erfahrung mit dem Labor
	\item Vergleich zu WebGoat
\end{itemize}
Insgesamt ist das Projekt SEED ein sehr umfangreiches Projekt mit vielen Angriffsszenarien. Die Angriffsszenarien sind in viele verschiedene Bereiche unterteilt, sodass die Möglichkeit besteht in allen verschiedenen Angriffsbereichen Erfahrungen zu sammeln. Da ich mit dieser Arbeit allerdings nur den Web Bereich bearbeite, ist dieser Bereich mit 6 verschiedenen Angriffsszenarien noch recht spärlich ausgestattet. Zum Vergleich bietet WebGoat von OWASP mit der Version 7.1 17 verschiedene Kategorien mit Angriffsszenarien, wobei jede Kategorie mehrere Aufgaben hat, die möglich sind, durchzugehen. 
\subsubsection{WebGoat}
Wie ich oben schon angemerkt habe würde ich das Projekt SEED im Bereich Web Security gerne mit der von OWASP entwickelten Webseite WebGoat vergleichen. OWASP steht für Open Web Application Security Project ist eine Non-Profit Organisation, welche die Sicherheit im World Wide Web verbessern möchte. Jeder kann der OWASP-Community beitreten. Die Community kommt aus verschiedenen Bereichen, so sind verschiedene Firmen, Universitäten, staatliche Agenturen und Einzelpersonen aus der ganzen Welt daran beteiligt die Projekte immer weiter voran zu treiben. \\
OWASP hat mehrere Projekte zu dem Thema Web Security. Ein sehr bekanntes Projekt ist die OWASP Top 10 Liste, welche die 10 meist kritischen Web Angriffe auflistet, mit einer kurzen Beschreibung und wie man sich davor schützen kann. Ein weiteres Projekt ist WebGoat. WebGoat ist eine Webseite, in der mit Absicht verschiedene Sicherheitslücken eingebaut wurden. Auf dieser Webseite gibt es sehr viele Aufgaben, die durchgegangen werden können. Zu den Aufgaben gibt es eine Story, die erklärt, was gemacht werden muss. Außerdem gibt es die Möglichkeit den Source Code anzusehen, sowie was das Ziel der jeweiligen Aufgabe ist. Sollte jemand nicht weiter kommen, besteht die Möglichkeit Hinweise zu bekommen was am besten gemacht werden sollte. Am Ende kann die eigene Lösung mit der Musterlösung verglichen werden. Diese kann auch verwendet werden, sollte jemand trotz der Hinweise nicht weiter kommen. Die Lösung ist eine Step to Step Anleitung was gemacht werden muss. Um die einzelnen Schritte zu verdeutlichen, wird jeder einzelne Schritt mit Screenshots verdeutlicht.

\subsubsection{Vergleich SEED und WebGoat}
SEED besitzt lediglich 6 verschiedene Angriffsszenarien. Jedes Angriffsszenario ist zwar in verschiedenen Aufgaben unterteilt, welche gut beschrieben sind und somit immer klar ist, was in der konkreten Aufgabe gemacht werden muss. Somit wird deutlich, was mit dem Angriff alles möglich ist. Jedoch fehlt das Ergebnis der einzelnen Aufgaben, sodass derjenige, der die Aufgabe durchführt, nie sicher sein kann, dass das was er gemacht hat der Aufgabe entspricht. Besonders schwierig wird es, wenn die durchführende Person den Angriff zum ersten Mal hört und ihn mit dem Labor lernen möchte. Zu jeder Aufgabe gibt es, im Gegensatz zu WebGoat, ein Video, welches erklärt, wie der Angriff funktioniert und wie es möglich ist, sich vor diesem Angriff zu schützen. \\
Auch ist der Aufbau von SEED und WebGoat unterschiedlich. Bei SEED muss eine ganze VM runter geladen werden und für jede Aufgabe muss in dieser VM extra Anpassungen getroffen werden, wie das Konfigurieren des Apache Servers oder des DNS Servers. Bei WebGoat muss die executable Jar-Datei runter geladen werden, sowie andere Tools wie zum Beispiel ein Tool zum Abfangen von HTTPs, welche bei SEED direkt in der VM enthalten sind. \\
Außerdem bietet SEED mehr das Gefühl, dass eine wirkliche Webseite angegriffen wird, als es bei WebGoat der Fall ist, da bei SEED jedes Labor eine eigene URL hat und dementsprechend jedes Labor eine eigene Seite besitzt. Bei WebGoat besitzen alle Angriffe die selbe URL und sind auf der selben Seite, was den Vorteil hat, dass zwischen den einzelnen Aufgaben einfacher gewechselt werden kann. Jedoch hat es den Nachteil, dass das einzelne Angriffsszenario nicht so wichtig erscheint und mehr gezwungener untergebracht ist, ohne eine wirkliche \enquote{Story}, wie es bei SEED der Fall ist. \\
Allerdings sind meiner Meinung nach die Aufgaben bei WebGoat besser und vielseitiger, da die Aufgaben von leicht zu schwer werden und immer die Möglichkeit besteht, die abgeschlossenen Aufgaben zu überprüfen und zu schauen, ob die eigenen Lösungen den gewünschten Ergebnissen entspricht. Außerdem gibt es Angriffe, welche nicht so bekannt sind und somit selbst wenn etwas Vorkenntnisse in dem Bereich Web Security bereits bestehen, immer noch die Möglichkeit besteht komplett neue Angriffe kennenzulernen. 

 