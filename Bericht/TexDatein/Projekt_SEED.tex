\chapter{Das Projekt SEED}
SEED ist ein Projekt der Universität Syracuse. Dieses Projekt soll einem helfen verschiedene Angriffsszenarien auszuprobieren. Damit alles legal ausgetestet werden kann, ist dieses Projekt in einer virtuellen Maschine. Somit werden alle Seiten, auf denen getestet werden soll auf einen lokalen Speicher in der virtuellen Maschine geleitet, ohne das es für den Angreifer einen unterschied macht. \\
Insgesamt gibt es 6 verschiedene Kategorien von Laboren. Jede Kategorie hat verschiedene Labore, die zu der entsprechenden Kategorie gehören.
\subsubsection{Software Security Labs}
Diese Labore zeigen generelle Angriffsszenarien auf einer Software. 
\subsubsection{Network Security Labs}
\subsubsection{Web Security Labs}
\subsubsection{System Security Labs}
\subsubsection{Cryptography Labs}
\subsubsection{Mobile Security Labs}
\section{Aufbau der Labore}
Der Aufbau der Labore ist einheitlich aufgebaut. Zu jedem Angriffsszenario gibt es eine grobe Erklärung, was dieser Angriff ist. Zusätzlich gibt es eine Aufgabe zu jedem Labor. Diese Aufgaben dauern unterschiedlich lange und sind auch je nach Angriffsszenario unterschiedlich schwer. Die Aufgaben bieten zunächst einen groben Überblick, was das Angriffsszenario ist und was in dieser Aufgabe gemacht wird. Als nächstes folgt eine Anleitung, wie das Labor, also die virtuelle Maschine, einzurichten ist. Manchmal reicht es die virtuelle Maschine laufen zu lassen. Allerdings müssen bei manchen Laboren die Konfiguration des Apache Servers geändert werden, damit dieser HTTP Anfragen nicht an das WWW raus schickt, sondern lokal in der virtuellen Maschine bleibt. Zusätzlich brauchen manche Labore Dateien, die extra in die virtuelle Maschine rein geladen werden müssen beziehungsweise Programme die installiert werden müssen, da sie in der Standard virtuellen Maschine nicht vorhanden sind. Da die Labore zum Beispiel Wissen über Programme oder über Programmiersprachen wie SQL voraussetzen, gibt es bevor das Labor los geht eine Auflistung der Sachen, welches Wissen für das Labor vorhanden sein muss. \\
Nun folgt das eigentliche Labor. Dieses ist so aufgebaut, dass es erst immer eine Erklärung gibt, was die Aufgabe machen soll, und was damit alles angestellt werden kann. Zusätzlich werden Informationen bekannt gegeben, die der Angreifer haben muss. Diese Informationen können zum Beispiel die Anmeldedaten einer Person sein um in die Anwendung rein zu kommen, oder auch der generelle Aufbau der Webseite. Anschließend folgt eine Erklärung was in diesem Labor gemacht werden muss. Diese Erklärungen können zum Teil sehr detailliert sein, sodass nur die Anweisungen befolgt werden müssen und kein beziehungsweise wenig eigenes Wissen hinzugefügt werden muss. Je mehr Aufgaben in dem Labor gemacht werden, desto schwieriger werden die Aufgaben. Somit muss besonders zum Ende hin mehr eigenes Wissen hinzugefügt werden. 
\section{Erfahrung zu dem Projekt SEED}
Insgesamt ist das Projekt SEED ein sehr umfangreiches Projekt mit vielen Angriffsszenarien. Die Angriffsszenarien sind in viele verschiedene Bereiche unterteilt, sodass die Möglichkeit besteht in allen verschiedenen Angriffsbereichen Erfahrungen zu sammeln. Da ich mit dieser Arbeit allerdings nur den Web Bereich bearbeite, ist dieser Bereich mit 6 verschiedenen Angriffsszenarien noch recht spärlich ausgestattet. Zum Vergleich bietet WebGoat von OWASP 