\chapter{Das Projekt SEED}
SEED ist ein Projekt der Universität Syracuse. Dieses Projekt soll einem helfen verschiedene Angriffsszenarien auszuprobieren. Damit alles legal ausgetestet werden kann, ist dieses Projekt in einer virtuellen Maschine. Somit werden alle Seiten, auf denen getestet werden soll auf einen lokalen Speicher in der virtuellen Maschine geleitet, ohne das es für den Angreifer einen unterschied macht. \\
Insgesamt gibt es 6 verschiedene Kategorien von Laboren. Jede Kategorie hat verschiedene Labore, die zu der entsprechenden Kategorie gehören.
\subsubsection{Software Security Labs}
Diese Labore zeigen generelle Angriffsszenarien auf einer Software. 
\subsubsection{Network Security Labs}
\subsubsection{Web Security Labs}
\subsubsection{System Security Labs}
\subsubsection{Cryptography Labs}
\subsubsection{Mobile Security Labs}
\section{Aufbau der Labore}