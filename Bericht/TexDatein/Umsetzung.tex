\chapter{Umsetzung}
\section{Aufbau der Webseite}
In dem Labor wird sich nicht so sehr um die Passwörter gekümmert, sondern mehr um die Session-IDs. Es soll damit klar werden, was eine Session-ID alles ist, wie es möglich ist, eine Session-ID zu erhalten und was ein Angreifer mit einer Session-ID anstellen kann. Hierzu wird die Sicherheit der Session-ID immer weiter verstärkt. Das Layout der Webseite ist schlicht gehalten, um nicht zu viel von der eigentlichen Ziel der Aufgabe abzulenken. \\
Mit den weiterkommen der Aufgaben in diesem Labor, wird der Server immer wieder verändert. Dabei muss kein Code explizit geschrieben werden, sondern vielmehr vorhandene Kommentare entkommentiert werden und die alten Zeilen kommentiert werden. Der Ausführung der Aufgaben soll nicht an dem Programmierkenntnissen der Anwender scheitern. \\
Es wird eine lokale Webseite geschrieben, die jedoch mithilfe des Apache Servers eine eigene Webadresse bekommt. Damit kann der Anwender nach Veränderung der Apache Konfiguration die Webseite wie über eine normale URL wie bei einer realen Webseite erreichen. Da für die Webseite ein NodeJS Server geschrieben wurde, hat dieser die Adresse \enquote{localhost}. Diese kann mithilfe des Apache Parameters \enquote{ProxyPass} auf eine eigene definierte URL umgeleitet werden. \\
Als Server wurde wie gerade schon beschrieben NodeJS verwendet. NodeJS ist ein Server in einer JavaScript Laufzeitumgebung, welcher für GoogleChrome entwickelt wurde. Mit NodeJS lassen sich besonders gut viele Netzwerkverbindungen bereitstellen. Neben Standardmodulen können auch weitere Module einfach hinzugefügt werden. Auch für Session-IDs gibt es ein Modul, dass dieses handhabt. Allerdings wird in diesem Labor erst gezeigt, wie von Hand mit Session IDs umgegangen wird. 
\subsection{Senden der Session-ID über URL}
Als aller erstes wird nach erfolgreicher Anmeldung der Client auf eine neue URL umgeleitet. Der einzige Unterschied von der alten URL zu der neuen URL ist, dass an dem Ende der neuen URL ein weiterer Parameter mit der Session-ID hinzugefügt wird. Mit dieser Session-ID authentifiziert sich der Benutzer dem Server gegenüber. \\
In dieser Aufgabe wird beschrieben, was passieren kann, wenn die Session-ID in der URL ist. \\
Sollte beispielsweise A sich auf der Seite authentifizieren und danach die URL mit der Session-ID an B schicken, so ist sowohl A als auch B auf der Webseite als B authentifiziert. Somit kann B sämtliche Aktionen auf der Webseite durchführen. Sollte es zum Beispiel möglich sein, auf der Webseite Buchungen durch zu führen, oder Bestellungen aufgeben, so ist B in der Lage dies zu tun und zwar alles im Namen von A. Somit kann A einen erheblichen Schaden zugefügt bekommen, ohne dass A es direkt mitbekommt. \\
Dabei trifft dem Opfer nicht eine große Schuld. Es könnte darauf achten, dass in einer URL eine Session-ID enthalten ist. Dies könnte bei besonders langen URLS einen sehr großen Aufwand bedeuten. Auch ist es nicht immer direkt klar, dass eine Session-ID über die URL versendet wird, und welcher Parameter die Session-ID ist, besonders wenn die URL ähnliche Parameternamen verwendet, oder Parameternamen die keinen Aufschluss auf deren Inhalt preisgeben. Ein weiteres Problem, welches viel schlimmer ist, ist dass der Benutzer, welcher den Link versendet, das Wissen über die Session-ID besitzen muss. Das heißt der Anwender muss Session-IDs kennen, wissen, dass diese über eine URL versendet werden und zusätzlich noch wissen, was mit einer Session-ID angerichtet werden kann, damit der Benutzer es für nötig hält, darauf zu achten. Dieses Wissen übersteigt bei Weitem das Wissen eines normalen Anwenders. Selbst wenn ein Benutzer dieses Wissen hat, kann es schnell passieren, dass der Benutzer einen Link weiter gibt, weil er gerade nicht darauf achtet. \\
Aus diesem Grund ist es die Aufgabe des Servers darauf zu achten, dass lediglich der Benutzer, dessen Session-ID es ist, sich mit der Session-ID authentifizieren kann. Somit darf niemals eine Session-ID über eine URL gesendet werden. \\
In dieser Aufgabe wird genau dieses Verhalten beschrieben. Ein Benutzer A meldet sich an einem System an und schickt danach die URL an eine andere Person. Diese öffnet die URL und erhält Informationen, die ihm eigentlich nicht zugänglich sein sollten.
\subsection{Senden der Session-ID als einfaches Cookie}
Wie in dem letzten Absatz beschrieben, darf eine Session-ID nicht über eine URL versendet werden, da sie dazu viel zu viele Angriffsstellen bietet. Wenn wirklich eine Session-ID versendet werden soll, bietet sich dazu die Verwendung eines Cookie als gute Alternative zu den versenden der Session-ID in der URL. Anstatt die URL zu verändern, wird ein Cookie bei dem Anwender gesetzt und wenn der Anwender eine Anfrage an dem Server schickt, wird zur Authentifizierung der Wert des Cookies abgefragt. Sollte nun die URL an eine andere Person gelangen, ist diese nicht direkt authentifiziert, sondern muss sich erst selbst authentifizieren, bevor diese den eigentlichen Inhalt sehen kann. Dieses Problem wäre damit gelöst. Allerdings ist ein einfaches setzen eines Cookies nicht genügend Sicherheit für ein so wichtigen Wert. \\
Einfache Cookies können von jedem ausgelesen, verändert und abgefangen werden. Aus diesem Grund wird in den nächsten Aufgaben erklärt, wie das alles verhindert werden kann und wie der Server sicher sein kann, dass keine andere Person in der Lage sein kann, den Wert der Session-ID zu erkennen.
\begin{itemize}
	\item Architektur der Webseite
	\item Einbau der Sicherheitslücke
\end{itemize}
\section{Durchführung des Labors}
\begin{itemize}
	\item Anleitung zur Durchführung 
	\item Schwierigkeit der Aufgaben
	
\end{itemize}
\section{besondere Hinweise}
\begin{itemize}
	\item Schwierigkeiten beim Erstellen + Lösung
\end{itemize}